\documentclass[11pt,a4paper]{report}

\usepackage[a4paper, margin=1in]{geometry}
\usepackage{polski}
\usepackage[utf8]{inputenc}



\title{\Huge GridGraph - Specyfikacja}
\author{Skoczek Mateusz, Jędrzejewski Sebastian}
\date{\today}



\begin{document}

    \maketitle
    

    \begin{abstract}
        Dokument zawiera specyfikację funkcjonalną oraz implementacyjną dotyczącą projektu \textsl{GridGraph}
    \end{abstract}


    \tableofcontents
    \thispagestyle{empty}


    \newpage
    \chapter{Specyfikacja funkcjonalna}


    \newpage
    \section{Cel projektu}
    Program \verb|GridGraph| ma na celu wygenerowanie oraz zapis do pliku (lub na standardowe wyjście) grafu siatkowego o podanych paramentrach lub wczytanie grafu z pliku (lub ze standardowego wejścia) i sprawdzenie wybranych jego parametrów. Program działa w trybie wsadowym. Grafy są przedstawiane w plikach w postaci listy sąsiedztwa.

    \newpage
    \section{Opis funkcji}
    Program może działać w dwóch trybach: zapisu (\verb|write|) i czytania (\verb|read|).\\
    \\
    \\
    W trybie zapisu program generuje graf o określonej przez użytkownika szerokości (ilości kolumn) (\verb|width|), wysokości (ilości wierszy) (\verb|height|), minimalnej (\verb|edge_weight_min|) i maksymalnej (\verb|edge_weight_max|) wagi krawędzi oraz minimalnej (\verb|edge_count_min|) i maksymalnej (\verb|edge_count_max|) ilości krawędzi wychodzących z jednego wierzchołka, a następnie zapisuje go w formie listy sąsiedztwa do pliku określonego przez użytkownika (lub wypisuje na standardowe wyjście).\\
    \\
    Jeżeli graf zostanie pomyślnie zapisany do pliku (lub wypisany na standardowe wyjście), program zwróci \verb|0|. W przeciwnym wypadku zostanie wyświetlony komunikat błędu, a program zwróci \verb|1|.\\
    \\
    \\
    W trybie czytania program wczytuje graf zapisany (w formie listy sąsiedztwa) w określonym przez użytkownika pliku (lub czyta ze standardowego wejścia), a następnie sprawdza określone przez użytkownika właściwości grafu:
    \begin{itemize}
        \item Spójność grafu (\verb|connectivity|)
        \item Najkrótsza ścieżka z węzła A do innych węzłów (\verb|shortest_path_a|) lub do określonego węzła B (\verb|shortest_path_a| oraz \verb|shortest_path_b|)
    \end{itemize}
    Jeżeli graf został wczytany oraz sprawdzony pomyślnie, zostanie wyświetlony wynik sprawdzania…\\
    \\
    Przykład (graf spójny, ścieżka istnieje):\\
    \verb|Connectivity: connected|\\
    \verb|Shortest path from 0 to 10 (weight): 0-3-4-6-9-10 (0.778)|\\
    \\
    Przykład (graf niespójny, ścieżka nie istnieje):\\
    \verb|Connectivity: disconnected|\\
    \verb|Shortest path from 0 to 10 (weight): path does not exist|\\
    \\
    …a następnie program zwróci \verb|0|. W przeciwnym wypadku zostanie wyświetlony komunikat błędu, a program zwróci \verb|1|

    \newpage
    \section{Opis wywołania}
    \subsection{Tryb zapisu}
    Wywołanie:\\
    \\
    \verb|./gridgraph --write/-w [argumenty]|\\
    \\
    Argumenty:
    \begin{itemize}
        \begin{samepage}
            \item \verb|--width/-xw| (Szerokość grafu - liczba kolumn)
            \begin{description}
                \item[Typ:] Liczba naturalna
                \item[Zakres:] $>$ 0 
                \item[Wymagany:] TAK
            \end{description}
        \end{samepage}
        \begin{samepage}
            \item \verb|--height/-xh| (Wysokość grafu - liczba wierszy)
            \begin{description}
                \item[Typ:] Liczba naturalna
                \item[Zakres:] $>$ 0 
                \item[Wymagany:] TAK
            \end{description}
        \end{samepage}
        \begin{samepage}
            \item \verb|--edge_weight_min/-Wmin| (Minimalna waga pojedyńczej krawędzi)
            \begin{description}
                \item[Typ:] Liczba rzeczywista
                \item[Zakres:] $<$0, edge\_weight\_max$>$
                \item[Wymagany:] NIE (domyślnie: 0)
            \end{description}
        \end{samepage}
        \begin{samepage}
            \item \verb|--edge_weight_max/-Wmax| (Maksymalna waga pojedyńczej krawędzi)
            \begin{description}
                \item[Typ:] Liczba rzeczywista
                \item[Zakres:] $<$edge\_weight\_min, 1$>$
                \item[Wymagany:] NIE (domyślnie: 1)
            \end{description}
        \end{samepage}
        \begin{samepage}
            \item \verb|--edge_count_min/-Cmin| (Minimalna liczba krawędzi wychodzących z jednego wierzchołka)\footnote{Program będzie dążył do utworzenia co najmniej edge\_count\_min krawędzi, ale nie może tego zagwarantować. Nie jest możliwe wygenerowanie więcej niż 2 krawędzi dla wierzchołków w narożnikach oraz więcej niż 3 dla wierzchołków bocznych. Nie jest możliwe także utworzenie krawędzi, jeżeli wszystkie wierzchołki wokół osiągnęły już swoją nominalną (wylosowaną z podanego przedziału) liczbę krawędzi.}
            \begin{description}
                \item[Typ:] Liczba naturalna
                \item[Zakres:] $<$0, edge\_count\_max$>$
                \item[Wymagany:] NIE (domyślnie: 0)
            \end{description}
        \end{samepage}
        \begin{samepage}
            \item \verb|--edge_count_max/-Cmax| (Maksymalna liczba krawędzi wychodzących z jednego wierzchołka)
            \begin{description}
                \item[Typ:] Liczba naturalna
                \item[Zakres:] $<$edge\_count\_min, 4$>$
                \item[Wymagany:] NIE (domyślnie: 4)
            \end{description}
        \end{samepage}
        \begin{samepage}
            \item \verb|--file/-f| (Plik w którym ma zostać zapisany graf)
            \begin{description}
                \item[Typ:] Ścieżka do pliku
                \item[Zakres:] -
                \item[Wymagany:] NIE (domyślnie: standardowe wyjście)
            \end{description}
        \end{samepage}
    \end{itemize}
    Przykład:\\
    \verb|./gridgraph -w -xw 6 -xh 6 --Wmin 0.65 --Wmax 0.2 -Cmax 3 -f "/home/user/graph”|
    \\
    \\
    Powyższy przykład ilustruje wywołanie programu, który generuje graf o 6 kolumnach i 6 wierszach, z wagami krawędzi mieszczącymi się w przedziale od 0.2 do 0.65, gdzie minimalna ilość krawędzi wychodzących z wierzchołka to 0, a maksymalna ilość krawędzi to 3. Program zapisuje graf w odpowiednim formacie do pliku o nazwie \verb|graph| znajdującego się w \verb|/home/user|.




    
\end{document}